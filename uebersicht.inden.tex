# section: Übersicht

# frame:
  # frametitle: Übersicht

  Heute geht es um vier prominente Sankt Petersburger Mathematiker:
  # itemize:
    * Pafnutij Lwowitsch \textbf{Tschebyschow}
    * Andrej Andrejewitsch \textbf{Markow}
    * Aleksandr Michajlowitsch \textbf{Ljapunow}
    * Aleksej Nikolajewitsch \textbf{Krylow}

  Vielleicht nächstes Mal: Moskauer Mathematiker, z.B.
  Lusin, Aleksandrow, Pontrjagin, Kotelnikow, Kolmogorow, Tikhonow, Arnold.

# frame:
  # frametitle: Übersicht

  # center:
    \includegraphics[height = 7.5 cm]{maps/st_petersburg_location}

# frame:
  # frametitle: Übersicht

  # tikzpicture:
    \footnotesize
    \begin{scope}[x = 8 cm, y = -0.5 cm]
      \draw (18.25, -1.0) node[above] {1825} -- +(0, 7.5);
      \draw (18.50, -1.0) node[above] {1850} -- +(0, 7.5);
      \draw (18.75, -1.0) node[above] {1875} -- +(0, 7.5);
      \draw (19.00, -1.0) node[above] {1900} -- +(0, 7.5);
      \draw (19.25, -1.0) node[above] {1925} -- +(0, 7.5);
      \draw (19.50, -1.0) node[above] {1950} -- +(0, 7.5);
      \draw[fill, rounded corners = 0.08 cm] (18.21, 0.0) rectangle (18.94, 1.0);
      \draw[fill, rounded corners = 0.08 cm] (18.56, 1.5) rectangle (19.22, 2.5);
      \draw[fill, rounded corners = 0.08 cm] (18.57, 3.0) rectangle (19.18, 4.0);
      \draw[fill, rounded corners = 0.08 cm] (18.63, 4.5) rectangle (19.45, 5.5);
      \node[right, white] at (18.21, 0.5) {Tschebyschow};
      \node[right, white] at (18.56, 2.0) {Markow};
      \node[right, white] at (18.57, 3.5) {Ljapunow};
      \node[right, white] at (18.63, 5.0) {Krylow};
    \end{scope}
