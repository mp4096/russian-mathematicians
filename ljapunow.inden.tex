# section: Aleksandr Ljapunow

# frame:
  # frametitle: Vita
  # columns:
    # column {0.4 \textwidth}:
      \includegraphics[width = \linewidth]{img/ljapunow}
    # column {0.6 \textwidth}:
      $*$ 6. Juni 1857 in Jaroslawl

      # vspace: 12 pt

      $\dagger$ 3. November 1918 in Odessa

# frame:
  # frametitle: Vita

  # center:
    \includegraphics[height = 7.5 cm]{maps/ljapunow}

# frame:
  # frametitle: Vita
    # itemize:
      \item[1870 (13)] Umzug nach Nischni Nowgorod
      \item[1876 (19)] Studienbeginn an der Kaiserlichen St. Petersburger Universität
      \item[1882 (25)] Studienabschluss
      \item[1885 (28)] Privatdozent an der Kaiserlichen St. Petersburger Universität;
      \item[{}] Umzug nach Kharkow;
      \item[{}] Privatdozent an der Kaiserlichen Kharkower Universität

# frame:
  # frametitle: Vita
    # itemize:
      \item[1892 (35)] Verteidigung der Dissertation \glqq Eine allgemeine Aufgabe zur Stabilität einer Bewegung\grqq
      \item[1893 (36)] Ordinarius an der Kaiserlichen Kharkower Universität
      \item[1901 (44)] Ordentliches Mitglied der St. Petersburger Akademie der Wissenschaften
      \item[1902 (45)] Umzug nach St. Petersburg
      \item[1917 (60)] Umzug nach Odessa

# frame:
  # frametitle: Ljapunows Dissertation
  \emph{%
    \glqq
    Die Vorgehensweise, die normalerweise benutzt wird, besteht darin, dass in den zu
    untersuchenden Differentialgleichungen alle Terme, die höher als die erste Dimension sind,
    verworfen werden, und anstatt der ursprünglichen die so
    erhaltenen linearen Differentialgleichungen betrachtet werden.%
    \grqq
  }

# frame:
  # frametitle: Ljapunows Dissertation
  \emph{%
    \glqq
    Freilich, diese Vorgehensweise bringt eine wesentliche Vereinfachung ein. [\ldots]
    Jedoch wird keine Gültigkeit dieser Vereinfachung \emph{a priori} belegt,
    denn sie führt zum Ersetzen des ürsprünglichen Problems mit einem anderen,
    mit dem es überhaupt keinen Zusammenhang haben kann. In jedem Fall ist es offensichtlich,
    dass falls die Lösung des neuen Problems eine Antwort auf das alte geben kann,
    kann das nur unter bestimmten Bedingungen sein, und diese werden in der Regel nicht angegeben.%
    \grqq
  }

# frame:
  # frametitle: Ljapunows Dissertation
  \emph{%
    \glqq
    Die Aufgabe, die ich mir gestellt habe, als ich die vorliegende Untersuchung vorgenommen habe,
    kann folgendermaßen formuliert werden: zeige diejenigen Fälle, in denen die erste Näherung
    tatsächlich die Frage nach der Stabilität beantwortet, und gebe irgendwelche Methoden,
    die [die Frage nach der Stabilität] zumindest in manchen der Fälle,
    wenn die erste Näherung kein Urteil über die Stabilität zulässt,
    diese Frage beantworten würden.%
    \grqq
  }

# frame:
  # frametitle: Ljapunows Vermächtnis
  # itemize:
    * Ljapunow-Bedingung
    * Ljapunow-Exponente
    * \textbf{Ljapunow-Fraktal}
    * Ljapunow-Gleichung
    * Ljapunow-Stabilität
    * Ljapunow-Ungleichung
    * Ljapunow-Zeit
    * Zentraler Grenzwertsatz von Ljapunow

# frame:
  # frametitle: Ljapunow-Fraktal
  # center:
    \includegraphics[height = 7.5 cm]{img/fraktal}
