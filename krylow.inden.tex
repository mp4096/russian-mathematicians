# section: Aleksej Krylow

# frame:
  # frametitle: Vita
  # columns:
    # column {0.4 \textwidth}:
      \includegraphics[width = 0.7 \linewidth]{img/krylow}
    # column {0.6 \textwidth}:
      $*$ 15. August 1863 in Krylowo

      # vspace: 12 pt

      $\dagger$ 26. Oktober 1945 in St. Petersburg

# frame:
  # frametitle: Vita

  # center:
    \includegraphics[height = 7.5 cm]{maps/krylow}

# frame:
  # frametitle: Fremdsprachenkenntnisse
  \emph{%
    \glqq
    Von all den Sachen, die du als Kind lernst, wirst du alles vergessen,
    außer dem, was du beruflich machen wirst, und außer Fremdsprachen,
    die du nur in Kindheit lernen kannst. Als Erwachsener kann man lesen und schreiben lernen,
    aber die Zunge, obwohl sie keine Knochen hat, kann man nicht brechen,
    und man spricht mit nischnigorodem Akzent,
    dabei sind die Fremdsprachenkenntnisse das Wichtigste im Leben.%
    \grqq
  }

# frame:
  # frametitle: Vita
    # itemize:
      \item[1872 (9)] Umzug nach Marseille
      \item[1874 (11)] Umzug nach Sewastopol
      \item[1875 (12)] Umzug nach Riga
      \item[1878 (15)] Aufnahme ins St. Petersburger Kadettenkorps der Marine
      \item[1884 (21)] Abschluss der Ausbildung
      \item[1887 (24)] Studienbeginn an der Nikolajew-Marineakademie
      \item[1890 (27)] Studienabschluss
      \item[1908 (45)] Hauptinspektor für Schiffbau
      \item[1910 (47)] Ordinarius an der Nikolajew-Marineakademie
      \item[{}] General der Kaiserlich Russischen Marine
      \item[1916 (53)] Ordentliches Mitglied der St. Petersburger Akademie der Wissenschaften

# frame:
  # frametitle: Krylows Schlagfertigkeit und Zitate
  # center:
    \includegraphics[width = \linewidth]{img/azneft}

# frame:
  # frametitle: Krylows Schlagfertigkeit und Zitate
  \resizebox{\linewidth}{!}{%
    \emph{\glqq Echter Ingenieur muss seinem Auge mehr als jeder Formel vertrauen.\grqq}%
  }
