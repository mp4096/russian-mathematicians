# section: Andrej Markow

# frame:
  # frametitle: Vita
  # columns:
    # column {0.4 \textwidth}:
      \includegraphics[width = \linewidth]{img/markow}
    # column {0.6 \textwidth}:
      $*$ 14. Juni 1856 in Rjasan

      # vspace: 12 pt

      $\dagger$ 20. Juli 1922 in Sankt Petersburg

# frame:
  # frametitle: Vita

  # center:
    \includegraphics[height = 7.5 cm]{maps/rjasan}

# frame:
  # frametitle: Vita
    # itemize:
      \item[1866 (10)] Umzug nach St. Petersburg
      \item[1874 (18)] Studiumbeginn an der Königlichen St. Petersburger Universität
      \item[1878 (22)] Studiumabschluss
      \item[1880 (24)] Privatdozent an der Königlichen St. Petersburger Universität
      \item[1886 (30)] Ordinarius an der Königlichen St. Petersburger Universität
      \item[1896 (40)] Ordentliches Mitglied der St. Petersburger Akademie der Wissenschaften

# frame:
  # frametitle: Markows Vermächtnis
  # itemize:
    * Dynamische Markow-Kompression
    * Hidden-Markow-Modell
    * Lempel-Ziv-Markow-Kette-Algorithmus (LZMA)
    * Markow-Ankunftsprozess
    * Markow-Clusteranalyse
    * Markow-Decke
    * Markow-Eigenschaft
    * Markow-Entscheidungsprozess
    * Markow-Kern
    * \textbf{Markow-Kette}
    * Markow-Ketten-Monte-Carlo
    * Markow-Kriterium
    * Markow-Logik-Netz
    * Markow-Matrix

# frame:
  # frametitle: Markows Vermächtnis
  # itemize:
    * Markow-Moment
    * Markow-Netzwerk
    * Markow-Parameter
    * Markow-perfektes-Gleichgewicht
    * Markow-Quelle
    * Markow-Spektrum
    * \textbf{Markow-Ungleichung}
    * Markow-Zahl
    * Markow-Zeit
    * Maximale-Entropie Markow-Modelle
    * Satz von Gauss-Markow
    * Satz von Markow-Krein
    * Teilweise beobachtbarer Markow-Entscheidungsprozess
    * Tschebyschow-Markow-Stieltjes-Ungleichung

# frame:
  # frametitle: Markow-Kette
  \centering
  # tikzpicture:
      \node[
        state,
        text = white,
        draw = none,
        fill = mDarkTeal,
        ]
        (s)
        {Sunny};
      \node[
        state,
        right = 3 cm of s,
        text = white,
        draw = none,
        fill = mDarkTeal,
        ]
        (r)
        {Rainy};

      \draw[
        every loop,
        auto = right,
        line width = 1mm,
        >=latex,
        draw = TolDarkBrown,
        fill = TolDarkBrown,
        ]
        (s) edge[bend right, auto = left] node {0.6} (r)
        (r) edge[bend right, auto = right] node {0.7} (s)
        (s) edge[loop above] node {0.4} (s)
        (r) edge[loop above] node {0.3} (r);

# frame:
  # frametitle: Markow-Ungleichung

  Markow veröffentlichte im Jahre 1889 die Ungleichung für $k = 1$;
  1892 verallgemeinerte sein Bruder Wladimir sie für höhere Ableitungen.

  # vspace: 12 pt

  Für ein Polynom $P(x)$ mit $\mathsf{deg}(P(x)) \leqslant n$ gilt
  # equation*:
    \max_{x \in [-1, 1]}
    \left|
      \frac{\mathsf{d}^k P(x)}{\mathsf{d} x^k}
    \right|
    \leqslant
    \prod_{i = 1}^{k}
    \frac{\left( n^2 - (i - 1)^2 \right)}{( 2i - 1 )}
    \max_{x \in [-1, 1]} |P(x)|
