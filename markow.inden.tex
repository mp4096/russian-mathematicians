# section: Andrej Markow

# frame:
  # frametitle: Vita
  # columns:
    # column {0.4 \textwidth}:
      \includegraphics[width = \linewidth]{img/markow}
    # column {0.6 \textwidth}:
      $*$ 14. Juni 1856 in Rjasan

      # vspace: 12 pt

      $\dagger$ 20. Juli 1922 in Sankt Petersburg

  \color{mDarkTeal!40!white}
  \tiny{Image source: \url{https://upload.wikimedia.org/wikipedia/commons/7/70/AAMarkov.jpg}}

# frame:
  # frametitle: Vita

  # center:
    \includegraphics[height = 7.5 cm]{maps/rjasan}

# frame:
  # frametitle: Vita
    # itemize:
      \item[1866 (10)] Umzug nach St. Petersburg
      \item[1874 (18)] Studienbeginn an der Kaiserlichen St. Petersburger Universität
      \item[1878 (22)] Studienabschluss
      \item[1880 (24)] Privatdozent an der Kaiserlichen St. Petersburger Universität
      \item[1886 (30)] Ordinarius an der Kaiserlichen St. Petersburger Universität
      \item[1896 (40)] Ordentliches Mitglied der St. Petersburger Akademie der Wissenschaften

# frame:
  # frametitle: Markows Vermächtnis
  # itemize:
    * Dynamische Markow-Kompression
    * Hidden-Markow-Modell
    * Lempel-Ziv-Markow-Kette-Algorithmus (LZMA)
    * Markow-Ankunftsprozess
    * Markow-Clusteranalyse
    * Markow-Decke
    * Markow-Eigenschaft
    * Markow-Entscheidungsprozess
    * Markow-Kern
    * \textbf{Markow-Kette}
    * Markow-Ketten-Monte-Carlo
    * Markow-Kriterium
    * Markow-Logik-Netz
    * Markow-Matrix

# frame:
  # frametitle: Markows Vermächtnis
  # itemize:
    * Markow-Moment
    * Markow-Netzwerk
    * Markow-Parameter
    * Markow-perfektes-Gleichgewicht
    * Markow-Quelle
    * Markow-Spektrum
    * \textbf{Markow-Ungleichung}
    * Markow-Zahl
    * Markow-Zeit
    * Maximale-Entropie Markow-Modelle
    * Satz von Gauss-Markow
    * Satz von Markow-Krein
    * Teilweise beobachtbarer Markow-Entscheidungsprozess
    * Tschebyschow-Markow-Stieltjes-Ungleichung

# frame:
  # frametitle: Markow-Ungleichung

  Markow veröffentlichte im Jahre 1889 die Ungleichung für $k = 1$;
  1892 verallgemeinerte sein Bruder Wladimir sie für höhere Ableitungen.

  # vspace: 12 pt

  Für ein Polynom $P(x)$ mit $\mathsf{deg}(P(x)) \leqslant n$ gilt
  # equation*:
    \max_{x \in [-1, 1]}
    \left|
      \frac{\mathsf{d}^k P(x)}{\mathsf{d} x^k}
    \right|
    \leqslant
    \prod_{i = 1}^{k}
    \frac{\left( n^2 - (i - 1)^2 \right)}{( 2i - 1 )}
    \max_{x \in [-1, 1]} |P(x)|

# frame:
  # frametitle: Markow-Kette
  \centering
  # tikzpicture:
    \node[
      state,
      text = white,
      draw = none,
      fill = mDarkTeal,
      ]
      (s)
      {Sonne};
    \node[
      state,
      right = 3 cm of s,
      text = white,
      draw = none,
      fill = mDarkTeal,
      ]
      (r)
      {Regen};

    \draw[
      every loop,
      auto = right,
      line width = 1mm,
      >=latex,
      draw = TolDarkBrown,
      fill = TolDarkBrown,
      ]
      (s) edge[bend right, auto = left] node {0.4} (r)
      (r) edge[bend right, auto = right] node {0.7} (s)
      (s) edge[loop above] node {0.6} (s)
      (r) edge[loop above] node {0.3} (r);

# frame:
  # frametitle: Markow-Ketten für Textgenerierung
  \texttt{%
    Sein Kugelschreiber ist rot. Mein Kugelschreiber ist blau.%
  }

  # itemize:
    * \texttt{("{}Sein", "Kugelschreiber")} $\rightarrow$ \texttt{"{}ist"}
    * \texttt{("Kugelschreiber", "{}ist")} $\rightarrow$ \texttt{"rot."}
    * \texttt{("{}ist", "rot.")} $\rightarrow$ \texttt{"Mein"}
    * \texttt{("rot.", "Mein")} $\rightarrow$ \texttt{"Kugelschreiber"}
    * \texttt{("Mein", "Kugelschreiber")} $\rightarrow$ \texttt{"{}ist"}
    * \texttt{("Kugelschreiber", "{}ist")} $\rightarrow$ \texttt{"blau."}

# frame:
  # frametitle: Markow-Ketten für Textgenerierung
  # itemize:
    * \texttt{("{}ist", "rot.")} $\rightarrow$ \texttt{\{("Mein", 1.0)\}}
    * \texttt{("Kugelschreiber", "{}ist")} $\rightarrow$ \texttt{\{("rot.", 0.5), ("blau.", 0.5)\}}
    * \texttt{("Mein", "Kugelschreiber")} $\rightarrow$ \texttt{\{"{(}ist", 1.0)\}}
    * \texttt{("rot.", "Mein")} $\rightarrow$ \texttt{\{("Kugelschreiber", 1.0)\}}
    * \texttt{("{}Sein", "Kugelschreiber")} $\rightarrow$ \texttt{\{"{(}ist", 1.0)\}}

# frame:
  # frametitle: Markow-Ketten für Textgenerierung
  \emph{%
    \glqq
    \ldots [dem] Publikum ist schließlich interessant zu wissen,
    dass der Christusglaube das Abendland überlebt hat,
    in der Bayerischen Wirtschaft Präsident der Technischen Universität München,
    mittendrin! Was wir heute alleine in Weihenstephan auf den Dreieinigen Gott
    kommt es auf das Zusammenspiel aller Stimmen an, genauso wie in einer erfolgreichen Universität.
    Das gelingt freilich nur, wenn man sich gegenseitig hört. Wie oft und intensiv haben sie, hier wie
    dort, ihren eigenen Part geübt, um schließlich das Team zu bereichern! Loyalität im besten Sinne.%
    \grqq
  }

# frame:
  # frametitle: Markow-Ketten für Textgenerierung
  \emph{%
    \glqq
    Jeder Euro zählt in seiner Konsequenz von Martin Luther noch einmal zurück:
    Denn Richard Wagner hat es seit Haber und Bosch (1918 bzw. 1931)
    Nobelpreise für Durchbrüche in der Musik. Mit ihr werden wir Martin Luther sein
    und passt zu Garching, weil Naturwissenschaft und Technik immer wieder
    überraschend hoffnungsvollen gnädigen Welt!%
    \grqq
  }
